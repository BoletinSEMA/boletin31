
\section{Ilustraci�n de la portada}
\label{sec:imagenportada}
\begin{center}\large
\textbf{Francisco Orteg�n Gallego},\\[.35ex] coeditor del Bolet�n electr�nico de la SEMA \\[.7ex]
{\color{azulsema}\rule{.5\linewidth}{1pt}}\\[2.3ex]
\end{center}

La imagen que sirve para ilustrar el presente n�mero del Bolet�n electr�nico de la SEMA est� inspirada en otra imagen que encontr� en una revista, no recuerdo cu�l, hace m�s de veinte a�os. 

La conocida identidad $\frac{\pi^2}{6} = \sum_{n=1}^\infty \frac{1}{n^2}$
puede interpretarse como que la suma de las �reas de todos los cuadrados cuyos  lados respectivos miden igual al inverso de cada uno de los n�meros naturales es igual al �rea de un rect�ngulo, $R$, de base $\pi^2/6$ y altura la unidad. 

En esta imagen se han apilado los cincuenta primeros cuadrados de la serie sin solapamiento y sin sobresalir del rect�ngulo $R$. Por tanto, el �rea pintada de negro es menor que $1/50$ unidades de superficie, y es mayor que $1/51$.

Se puede observar que todav�a hay espacio para colocar algunos cuadrados m�s de la serie. 
Se pueden plantear entonces varias cuestiones: �Se podr�a rellenar el rect�ngulo $R$ con los infinitos cuadrados de lados respectivos los inversos de todos los n�meros naturales mediante alguna configuraci�n de este tipo? En caso afirmativo, 
si $a>1$ y $b\in\left[\frac{13}{8},\frac{\pi^2}{6}\right)$ son tales que $ab=\frac{\pi^2}{6}$, �es posible rellenar el rect�ngulo de base $b$ y altura $a$ con estos mismos cuadrados?

\def\rectangulouno(#1,#2)#3\encolor#4{%
         \coordinate (A) at (#1,#2);
         \coordinate (B) at (1, 1);
%\filldraw[fill=#4, draw=green!50!black] (A) rectangle ++(B);
\filldraw[fill=#4, draw=white] (A) rectangle ++(B);
         \coordinate [label=center:$#3$] (X) at ($ (A) + .5*(B) $);}
         
\def\rectangulonum(#1,#2)#3\encolor#4{%
         \coordinate (A) at ($ (#1,#2) $);
         \coordinate (B) at ($ (1./#3, 1./#3 ) $);
%\filldraw[fill=#4, draw=green!50!black] (A) rectangle ++(B);
\filldraw[fill=#4, draw=white] (A) rectangle ++(B);
         \coordinate [label=center:$\frac{1}{#3^2}$] (X) at ($ (A) + .5*(B) $);}
         
\def\rectangulorecta(#1,#2)#3(#4,#5)\encolor#6{%
         \coordinate (A) at ($ (#1,#2) $);
         \coordinate (B) at ($ (1./#3, 1./#3 ) $);
%\filldraw[fill=#6, draw=green!50!black] (A) rectangle ++(B);
\filldraw[fill=#6, draw=white] (A) rectangle ++(B);
         \coordinate  (X) at ($ (A) + .5*(B) $);
         {\color{black}
         \coordinate [label=right:$\!\!\frac{1}{#3^2}$] (Y) at (#4,#5);
         \draw (X) -- (Y);}}

\def\rectangulorectaup(#1,#2)#3(#4,#5)\encolor#6{%
         \coordinate (A) at ($ (#1,#2) $);
         \coordinate (B) at ($ (1./#3, 1./#3 ) $);
%\filldraw[fill=#6, draw=green!50!black] (A) rectangle ++(B);
\filldraw[fill=#6, draw=white] (A) rectangle ++(B);
         \coordinate  (X) at ($ (A) + .5*(B) $);
         {\color{black}
         \coordinate [label=above:$\!\!\frac{1}{#3^2}$] (Y) at (#4,#5);
         \draw (X) -- (Y);}}

%% dibujo de la portada
\begin{center}
\resizebox{.95\linewidth}{!}{%
\begin{tikzpicture}[scale=9.333,>=stealth, very thin]

{\color{white}
\filldraw[fill=black, draw=white] (0,0) rectangle ($ (pi^2/6,1) $);}

{\Huge\rectangulouno(0,0){1}\encolor{green!20!white}}
{\Huge\rectangulonum(1,0){2}\encolor{red!40!white}}
{\Huge\rectangulonum(1,1/2){3}\encolor{blue!30!white}}
{\Huge\rectangulonum(1+1/3,1/2){4}\encolor{yellow!80!white}}
{\Huge\rectangulonum(1+1/3,1/2+1/4){5}\encolor{green!90!white}}
{\huge\rectangulonum(1.,1/2+1/3){6}\encolor{red!40!white}}
{\huge\rectangulonum(1+1/6,1/2+1/3){7}\encolor{brown!70!white}}
{\Large\rectangulonum(1+1/2,0){8}\encolor{yellow!30!white}}
{\Large\rectangulonum(1+1/2,1/8){9}\encolor{green!40!white}}
{\Large\rectangulonum(1+1/2,1/8+1/9){10}\encolor{red!50!white}}
{\large\rectangulonum(1+1/2,0.33611){11}\encolor{blue!60!white}}
{\large\rectangulonum(1.53333333,0.75){12}\encolor{brown!40!white}}
{\large\rectangulonum(1.53333333,0.75 + 1/12){13}\encolor{green!20!white}}
{\rectangulonum(1.53333333,0.75 + 1/12 +1/13){14}\encolor{red!30!white}}
{\rectangulonum(1.5,0.33611 + 1/11){15}\encolor{blue!30!white}}
{\rectangulonum(1.5 + 1/15, 0.33611 + 1/11){16}\encolor{yellow!40!white}}
{\rectangulonum(1.33333333 + 1/4, 0.5 + 1/4 - 1/17){17}\encolor{green!50!white}}
{\rectangulonum(1.33333333 + 1/4, 0.5 + 1/4 - 1/17 - 1/18){18}\encolor{red!60!white}}
{\small\textcolor{white}{\rectangulonum(1.33333333 + 1/4, 0.5 + 1/4 - 1/17 - 1/18 -1/19){19}\encolor{blue!70!white}}}
{\small\textcolor{red}{\rectangulonum(1.33333333 + 1/4, 0.5 + 1/4 - 1/17 - 1/18 -1/19 -1/20){20}\encolor{yellow!80!white}}}
{\small\textcolor{red}{\rectangulonum(1.5 + 1/11,0.33611 +1/11 -1/21){21}\encolor{green!90!white}}}
{\small\textcolor{white}{\rectangulonum(1.16666667+1/7,0.95){22}\encolor{red!90!white}}}
{\footnotesize\rectangulonum(1.16666667+1/7+1/22,0.95){23}\encolor{blue!10!white}}
{\footnotesize\rectangulonum(1.16666667+1/7+1/22+1/23,0.95){24}\encolor{yellow!20!white}}
{\tiny\rectangulonum(1.16666667+1/7+1/22+1/23+1/24,0.95){25}\encolor{green!30!white}}
{\tiny\rectangulonum(1.16666667+1/7+1/22+1/23+1/24+1/25,0.95){26}\encolor{red!40!white}}
%
\rectangulorecta(1+1/3+1/4,1/8+1/9+1/10+1/11+1/16){27}(1.7,1/8+1/9+1/10+1/11+1/16)\encolor{blue!50!white}
\rectangulorecta(1+1/2+1/10,1/8+1/9){28}(1.7,1/8+1/9)\encolor{yellow!60!white}
\rectangulorecta(1+1/2+1/10,1/8+1/9+1/28){29}(1.7,1/8+1/9+1/18)\encolor{green!70!white}
\rectangulorecta(1+1/2+1/10,1/8+1/9+1/28+1/29){30}(1.7,1/8+1/9+1/18+1/19)\encolor{red!80!white}
\rectangulorecta(1+1/2+1/11,1/8+1/9+1/28+1/29+1/30){31}(1.7,1/8+1/9+3/18)\encolor{blue!90!white}
\rectangulorecta(1+1/2+1/9,1/8){32}(1.7,1/8)\encolor{red!15!white}
\rectangulorecta(1+1/2+1/9,1/8+1/32){33}(1.7,1/8+1/18)\encolor{blue!25!white}
\rectangulorecta(1+1/2+1/9,1/8+1/32+1/33){34}(1.8,1/8+1/12)\encolor{yellow!35!white}
\rectangulorecta(1+1/3+1/5+1/13,1/2+1/4+1/12){35}(1.74,1/2+1/4+1/10)\encolor{yellow!45!white}
\rectangulorecta(1+1/3+1/5+1/13,1/2+1/4+1/12+1/35){36}(1.67,1/2+1/4+1/7.6)\encolor{red!45!white}
\rectangulorecta(1+1/3+1/5+1/12,1/2+1/4){37}(1.7,1/2+1/4)\encolor{red!45!white}
\rectangulorecta(1+1/3+1/5+1/12,1/2+1/4+1/37){38}(1.8,1/2+1/4+1/28)\encolor{blue!55!white}
\rectangulorecta(1+1/3+1/5+1/12,1/2+1/4+1/37+1/38){39}(1.68,1/2+1/4+1/14)\encolor{green!65!white}
\rectangulorecta(1+1/3+1/5+1/13,1/2+1/4+1/12+1/35+1/36){40}(1.8,1/2+1/4+1/12+1/35+1/17)\encolor{blue!75!white}
\rectangulorecta(1+1/3+1/5+1/14,1/2+1/4+1/12+1/35+1/36+1/40){41}(1.68,1/2+1/4+1/12+4/35)\encolor{yellow!85!white}
\rectangulorecta(1+1/3+1/5+1/14,1/2+1/4+1/12+1/35+1/36+1/40+1/41){42}(1.74,1/2+1/4+1/12+5.5/35)\encolor{green!95!white}
\rectangulorecta(1+1/3+1/5+1/14,1/2+1/4+1/12+1/35+1/36+1/40+1/41+1/42){43}(1.68,1/2+1/4+1/12+6.5/35)\encolor{blue!12!white}
\rectangulorectaup(1+1/6+1/7,1/2+1/3+1/44+1/45+1/46+1/47){48}(1.455,1.1)\encolor{yellow!62!white}
\rectangulorectaup(1+1/6+1/7,1/2+1/3+1/44+1/45+1/46){47}(1.305,1.1)\encolor{red!52!white}
\rectangulorectaup(1+1/6+1/7,1/2+1/3+1/44+1/45){46}(1.255,1.1)\encolor{green!62!black}
\rectangulorectaup(1+1/6+1/7,1/2+1/3+1/44){45}(1.205,1.1)\encolor{yellow!52!black}
\rectangulorectaup(1+1/6+1/7,1/2+1/3){44}(1.15,1.1)\encolor{red!82!white}
{\small\textcolor{white}{\rectangulonum(1.16666667+1/7,0.95){22}\encolor{red!90!white}}}
\rectangulorectaup(1+1/6,1/2+1/3+1/7){49}(0.9,1.1)\encolor{yellow!72!white}
\rectangulorectaup(1+1/6+1/49,1/2+1/3+1/7){50}(1,1.1)\encolor{blue!52!white}

{\color{black}
\node (a) at (-.05,.5) {\Large$1$};
\node (b) at (1.644934067/2,-.05) {\Large$\pi^2/6$};
\draw [->] (a) -- (-.05,1);
\draw [->] (a) -- (-.05,0);
\draw [->] (b) -- (1.644934067,-.05);
\draw [->] (b) -- (0,-.05);}
\end{tikzpicture}
}

\mbox{}\\[.1ex]
\resizebox{.5\linewidth}{!}{\color{azulsema}\rule{.5\linewidth}{1pt}
{\large $\diamond$} {\huge $\diamond$} {\large $\diamond$} \rule{.5\linewidth}{1pt}}
\end{center}

